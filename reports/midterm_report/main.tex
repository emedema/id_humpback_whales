%%%%%%%%%%%%%%%%%%%%%%%%%%%%%%%%%%%%%%%%%%%%%%%%%%%%%%%%%%%%%%%%%%%%%%
% LaTeX Example: Project Report
%
% Source: http://www.howtotex.com
%
% Feel free to distribute this example, but please keep the referral
% to howtotex.com
% Date: March 2011 
% 
%%%%%%%%%%%%%%%%%%%%%%%%%%%%%%%%%%%%%%%%%%%%%%%%%%%%%%%%%%%%%%%%%%%%%%
% How to use writeLaTeX: 
%
% You edit the source code here on the left, and the preview on the
% right shows you the result within a few seconds.
%
% Bookmark this page and share the URL with your co-authors. They can
% edit at the same time!
%
% You can upload figures, bibliographies, custom classes and
% styles using the files menu.
%
% If you're new to LaTeX, the wikibook is a great place to start:
% http://en.wikibooks.org/wiki/LaTeX
%
%%%%%%%%%%%%%%%%%%%%%%%%%%%%%%%%%%%%%%%%%%%%%%%%%%%%%%%%%%%%%%%%%%%%%%
% Edit the title below to update the display in My Documents
%\title{Project Report}
%
%%% Preamble
\documentclass[paper=a4, fontsize=11pt]{scrartcl}
\usepackage[T1]{fontenc}
\usepackage{fourier}

\usepackage[english]{babel}															% English language/hyphenation
\usepackage[protrusion=true,expansion=true]{microtype}	
\usepackage{amsmath,amsfonts,amsthm} % Math packages
\usepackage[pdftex]{graphicx}	
\usepackage{parskip} % Sets space between paragraphs
\usepackage{url}
\usepackage{hyperref}
\usepackage{lineno}
\usepackage{color, soul}

% Adds colours to links
\hypersetup{
    colorlinks=true,
    linkcolor=magenta, % makes links to equations, figs, etc magenta
    urlcolor=blue, % makes url links blue
    citecolor = red % makes citation links red
}

\newcommand{\appropto}{\mathrel{\vcenter{
  \offinterlineskip\halign{\hfil$##$\cr
    \propto\cr\noalign{\kern2pt}\sim\cr\noalign{\kern-2pt}}}}}

%%% Custom sectioning
\usepackage{sectsty}
\allsectionsfont{\centering \normalfont\scshape}


%%% Custom headers/footers (fancyhdr package)
\usepackage{fancyhdr}
\pagestyle{fancyplain}
\fancyhead{}											% No page header
\fancyfoot[L]{}											% Empty 
\fancyfoot[C]{}											% Empty
\fancyfoot[R]{\thepage}									% Pagenumbering
\renewcommand{\headrulewidth}{0pt}			% Remove header underlines
\renewcommand{\footrulewidth}{0pt}				% Remove footer underlines
\setlength{\headheight}{13.6pt}


%%% Equation and float numbering
\numberwithin{equation}{section}		% Equationnumbering: section.eq#
%\numberwithin{figure}{section}			% Figurenumbering: section.fig#
\numberwithin{table}{section}				% Tablenumbering: section.tab#
\usepackage{gensymb}
\usepackage{relsize}
%%% Maketitle metadata
\newcommand{\horrule}[1]{\rule{\linewidth}{#1}} 	% Horizontal rule


%% To use line numbers 
\linenumbers

%% create a title page
\title{
		%\vspace{-1in} 	
		\usefont{OT1}{bch}{b}{n}
		\normalfont \normalsize \textsc{Queen's University} \\ [25pt]
		\horrule{0.5pt} \\[0.4cm]
		\huge Humpback Whale Identification Midterm Report \\
		\horrule{2pt} \\[0.5cm]
}
\author{
    \normalfont 
      CISC 867 - Deep Learning Project \\
    \normalfont
    Group Members: \\ 
    \normalsize
    Emily Medema (20340337) \\ 
    \normalsize
    Stephen McKeon () \\ 
    \normalsize
    Flourish Adebayo (20312488) \\
    October 2022 \\ [3pt]}
\date{\vspace{-5ex}}


%%% Begin document
\usepackage{graphicx}
\graphicspath{ {./images/} }
\begin{document}
%% remove the page number on the title page 
\pagenumbering{gobble}
%% need this line to add the title page you just created 
\maketitle

%% the section command gives a new section with the given header. 


%% go to a new page 
\newpage 
%% start the page numbering again 
\pagenumbering{arabic}

\section*{Introduction}\label{sec: intro}

Image classification is a booming area of interest in the Computer Vision and Machine Learning fields. Due to the rapid increase of image sharing after the popularity of social media and personal cameras (and later smart phones) \cite{}, there are a surplus of images to classify and analyze on the internet. There are different algorithms for image classification. The most common ones are deep learning and machine learning. Different models have distinct results in different problems. Image classification using traditional deep learning and machine learning algorithms has its advantages. 

%% TODO: go into pros and cons?

In fact, Deep Neural Networks (DNN) is exponentially growing in the field of Machine Learning (ML) and Deep Learning. Of the many DNN structures, Convolutional Neural Networks (CNN) are presently the main tool used for image analysis and classification purposes \cite{}. Comparison and evaluation of images using classification algorithms based on traditional machine learning and deep learning are of great significance for selecting algorithms to classify pictures. 

Due to the aforementioned popularity of cameras, many scientific studies are utilizing photography as a method of monitoring their projects. This usually results in a scientist having to analyze these images themselves, which can take many hours and a lot of technical knowledge. However, we can now use image classification models to perform these same tasks in a lot less time with comparable accuracy \cite{}. 

The Humpback Whale Identification Challenge is a Kaggle Competition created to aid whale conservation efforts with the creation of an algorithm to identify individual whales in images. After centuries of intense whaling, recovering whale populations still have a hard time adapting to warming oceans and struggle to compete every day with the industrial fishing industry for food. Scientists use photo surveillance systems to monitor whale activity and can use the shape of whales’ tails and unique markings to identify particular whales and analyze their movements.

%% TODO: this needs to be refined
We can use this data to compare and analyze popular methods of image classification. Specifically, we will be comparing a CNN, CNN augmented with transfer learning, and classical machine learning.

\section{Methodology}\label{sec: meth}

\subsection{Data Preprocessing}

The competition contains thousands of images of humpback whale flukes. Individual whales have been identified by researchers and given an identification (Id). The challenge is to predict the whale Id of images in the test set. What makes this such a challenge is that there are only a few examples for over 3,000 whale Ids.

Due to this, we will implement the following data augmentation methods in order to gain a larger set of images for each whale.

\section{Evaluation/Results}\label{sec: results}

\section{Bibliography and References}\label{sec: bibliography}

TODO: customize this, this is just here as a reminder as to how to do it

One of my favourite things is to use \LaTeX{} to format my bibliographies. To demonstrate this I will be using Bibtex, a reference management software that format's your references. I will also show you how to use JabRef to organize your references. In the same google drive that you downloaded this document from, download the bibliography file and load it into your overleaf. 


It is important to note the the bibliography does not print if you have not cited anything from it. Therefore I will cite the documents below to ensure the references appear on the last page. This is how you cite a document as an in text citation \cite{LuTao2021Daco}. There are many different bibliography styles available that you can look up. Notice the green text that appears inside the in text citations? That is the tag for the .bib entry. You need to come up with this when saving an entry in your .bib file. We will see an example of this shortly. 


If you compile the file, you will again notice that your new entry is not there. Make sure to cite the paper. 

 
\clearpage
\bibliography{references} 
\bibliographystyle{ieeetr}



%%% End document
\end{document}